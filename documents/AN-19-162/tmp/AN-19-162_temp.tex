%%%%%%%%%%%%%%%%%%%%%%%%%%%%%%%%%%%%%%%%%%%%%%%%%%%%%%%%%%%%%%%%%%%%
%
%   Style for CMS Computing / Physics Technical Design Reports
%
%   Lucas Taylor  4 Feb 2005,   Revised  12 Oct 2005
%
%%%%%%%%%%%%%%%%%%%%%%%%%%%%%%%%%%%%%%%%%%%%%%%%%%%%%%%%%%%%%%%%%%%%

%  the following line is edited by the tdr script to change or to pass
%  additional options:
\documentclass[11pt,twoside,a4paper,cmspaper]{cms-tdr}
\def\svnVersion{e94debb-D}\def\svnDate{2019/08/04}

%%%%%%%%%%%%%%%%%%%%%%%%%%%%%%%%%%%%%%%%%%%%%%%%%%%%%%%%%%%%%%%%%%%%

\begin{document}\cmsNoteHeader{AN-19-162}
%%%%%%%%%%%%%%%%%%%%%%%%%%%%%%%%%%%%%%%%%%%%%%%%%%%%%%%%%%%%%%%%%%%%
%
%  Common definitions
%
%  N.B. use of \providecommand rather than \newcommand means
%       that a definition is ignored if already specified
%
%                                              L. Taylor 18 Feb 2005
%%%%%%%%%%%%%%%%%%%%%%%%%%%%%%%%%%%%%%%%%%%%%%%%%%%%%%%%%%%%%%%%%%%%


%%%%%%%%%%%%%%%%%%%%%%%%%%%%%%%%%%%%%%%%%%%%%%%%%%%%%%%%%%%%%%%%%%%%
%
% Hyphenations (only need to add here if you get a nasty word break)
%
\hyphenation{had-ron-i-za-tion}
\hyphenation{cal-or-i-me-ter}
\hyphenation{de-vices}
%
% Hyphenations-end
% % Customizable fields and text areas start with % >> below.
% Lines starting with the comment character (%) are normally removed before release outside the collaboration, but not those comments ending lines

%%%%%%%%%%%%% local definitions %%%%%%%%%%%%%%%%%%%%%


%%%%%%%%%%%%%%%  Title page %%%%%%%%%%%%%%%%%%%%%%%%
\cmsNoteHeader{AN-19-162} % This is over-written in the CMS environment: useful as preprint no. for export versions
% >> Title: please make sure that the non-TeX equivalent is in PDFTitle below for papers. For PASs, PDFTitle can be used with plain TeX.
\title{Lepton flavor universality test at CMS with R(D*) measurement in the full leptonic tau final state}

% >> Authors
%Author is always "The CMS Collaboration" for PAS and papers, so author, etc, below will be ignored in those cases
%For multiple affiliations, create an address entry for the combination
%To mark authors as primary, use the \author* form
\address[inst1]{California Institute of Technology (US)}
\author*[inst1]{Olmo Cerri}

\date{\today}

% >> Abstract
% Abstract processing:
% 1. **DO NOT use \include or \input** to include the abstract: our abstract extractor will not search through other files than this one.
% 2. **DO NOT use %**                  to comment out sections of the abstract: the extractor will still grab those lines (and they won't be comments any longer!).
% 3. For PASs: **DO NOT use CMS tex macros.**...in the abstract: CDS MathJax processor used on the abstract doesn't understand them _and_ will only look within $$. The abstracts for papers are hand formatted so macros are okay.
\abstract{
   Study to perform a measurement of the branching fraction ratio $R(D^*) \equiv B(B_0 \to D^{*-}\tau^+\nu_\tau) / B(B_0 \to D^{*-}\mu^+\nu_\mu)$ with proton-proton CMS data coming from the 2018 B parking stream. The tau lepton is identified in the full leptonic decay mode $\tau^+ \to \mu^+ \nu_\tau \nu_\mu$.\\ The $R(D^*)$ ratio is sensitive to contributions from non-standard-model particles that violates lepton flavour universality. This parameter, measured at B factories and hadron collider, has been observed to have a tension with SM prediction of about 2 sigma.\\ A multidimensional fit to kinematic distributions of the reconstructed $B_0$ candidate decays from CMS data can provide a competitive measurement with state of art measurement and impact the experimental world average. This result, can be the first measurement of this quantity at CMS and general purpose hadron collider experiments.
}

% >> PDF Metadata
% Do not comment out the following hypersetup lines (metadata). They will disappear in NODRAFT mode and are needed by CDS.
% Also: make sure that the values of the metadata items are sensible and are in plain text with the possible exception of the PDFtitle for a PAS. Then you can use pure TeX symbols as if on a typewriter. Examples: $\sqrt{s}=13\TeV$ => $sqrt{s}=$ 13 TeV; 32\fbinv => 32 fb$^{-1}$
% No unescaped comment % characters.
% No curly braces {} except for TeX in the PDFtitle.
\hypersetup{%
pdfauthor={Olmo Cerri},%
pdftitle={Lepton flavor universality test at CMS with R(D*) measurement in the full leptonic tau final state},%
pdfsubject={CMS},%
pdfkeywords={CMS, physics, your topics}}

\maketitle

\section{Introduction}
The Standard Theory (SM) of particle physics predicts the three leptons generation to have the same coupling to gauge bosons. This simmetry, called lepton flavor universality (LFU), is an accidental symmetry and it is broken only by the Yukawa interactions.
Differences between the expected branching fraction of semileptonic decays into the three lepton families originate from the different masses of the charged leptons. Further deviations from LFU would be a signature of physics processes beyond the SM.
The consitency of the nature with this p[rediction can be tested in heavy mesons semi-leptonic decay.

State briefly what's the nalysis startegy.

\section{The CMS detector}

Which reco, which calibration gloabal tag and extractor

\section{Simulation}

\subsection{Signal simulation}

\subsection{Backgorunds simulation}

\section{Candidate selection}
Should mention all the cuts and relative efficiencies
\subsection{Final state obserbables}

\section{Backgrounds estimation}
\subsection{Simulation driven backgounds}
\subsection{Data driven backgrounds}

\section{Signals yeld determination}
\subsection{Uncertainties breakdown}

\section{Efficiencies estimation}

\section{Conclusion}

\subsection{Future prospects}


%%%%%%%%%%%%%%%%%%%%%%%%%%%%%%%%  Begin text %%%%%%%%%%%%%%%%%%%%%%%%%%%%%
%% **DO NOT REMOVE THE BIBLIOGRAPHY** which is located before the appendix.
%% You can take the text between here and the bibiliography as an example which you should replace with the actual text of your document.
%% If you include other TeX files, be sure to use "\input{filename}" rather than "\input filename".
%% The latter works for you, but our parser looks for the braces and will break when uploading the document.
%%%%%%%%%%%%%%%

% >> acknowledgments (for journal papers only)
% The latest version of the acknowledgments will be included from https://twiki.cern.ch/twiki/bin/viewauth/CMS/Internal/PubAcknow as of the date of submission. Modify to match either US or UK English spelling for centre/center, programme/program. For PRL use the short version, for JINST normally use the long version. All others take the middle length version other than exceptional cases.
\begin{acknowledgments}
This project has received funding from the European Research Council (ERC) under the European Union’s Horizon 2020 research and innovation program (grant agreement no 772369) and the United States Department of Energy, Office of High Energy Physics Research under Caltech Contract No. DE-SC0011925. This work was conducted at "\textit{iBanks}", the AI GPU cluster at Caltech. We acknowledge NVIDIA, SuperMicro and the Kavli Foundation for their support of "\textit{iBanks}".
\end{acknowledgments}

%% **DO NOT REMOVE BIBLIOGRAPHY**
\bibliography{auto_generated}   % will be created by the tdr script.
%% examples of appendices.
%\clearpage
%\appendix
%\section{Appendix name}
%%% DO NOT ADD \end{document}!
\end{document}

